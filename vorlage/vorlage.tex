\documentclass[a4paper,12pt]{article}                       
% Seitenränder
\usepackage[left=3cm,right=2cm,top=2cm,bottom=2cm,includehead]{geometry}
\usepackage{ngerman}
\usepackage[ngerman]{babel}
\usepackage[utf8]{inputenc}
\usepackage[hyperfootnotes=false]{hyperref}
\usepackage[nottoc]{tocbibind}
\usepackage{makeidx}
\usepackage[intoc]{nomencl}
\usepackage{fancyhdr}
\usepackage[round]{natbib}
\usepackage{amsmath}
\usepackage[labelfont=bf,aboveskip=1mm]{caption}
\usepackage{setspace}
\usepackage[bottom,multiple,hang,marginal]{footmisc}
\usepackage{graphicx}
\usepackage{tabularx}
\usepackage{longtable}
\usepackage{color}
\usepackage{enumitem}
\usepackage{listings}
\usepackage{zref}
\usepackage[toc,style=altlist,translate=false]{glossaries}
\usepackage{glossaries-babel}


%%%%%%%%%%%%%%%%%%%%%%% Definitionen bzgl. der Arbeit %%%%%%%%%%%%%%%%%%%%%%%
\def\myType{0}           % [0=Seminararbeit|1=Projektarbeit|2=Bachelorarbeit]

\def\myTopic{Thema der Arbeit}
\def\mySubTopic{Untertitel}
\def\myAutor{Dein Name}
\def\myCompany{Name des Unternehmens}
\def\myCompanyAddressStreet{Adresse des Unternehmens (Strasse)}
\def\myCompanyAddressCity{Adresse des Unternehmens (Ort)}
\def\myProf{Name des betreuenden Professors}
\def\myEndDate{Datum der Abgabe}


%%%%%%%%%%%%%%%%%%%% Folgende Angaben für: Seminararbeit %%%%%%%%%%%%%%%%%%%%
\def\myVorlesung{Name der Vorlesung}


%%%%%%%%%%%%%%%%%%%% Folgende Angaben für: Projektarbeit %%%%%%%%%%%%%%%%%%%%
\def\myProjNumber{Nummer der Projektarbeit}         % [1|2]
\def\myPraxPhase{Studienjahr}                       % [1|2|3]


%%%%%%%%%%%%%%%%%%%%%%%% Eigene Farbwerte definieren %%%%%%%%%%%%%%%%%%%%%%%%
\definecolor{boxgray}{gray}{0.9}         % Hintergrundfarbe für Zitatboxen
\definecolor{commentgray}{gray}{0.5}     % Grau für Kommentare in Quelltexten
\definecolor{darkgreen}{rgb}{0,.5,0}     % Grün für Strings in Quelltexten

%%%%%%%%%%%%%%%%%%%%%%%% Eigene Kommandos definieren %%%%%%%%%%%%%%%%%%%%%%%%
% Definition von \gqq{#1: text}: Text in Anführungszeichen
\newcommand{\gqq}[1]{\glqq #1\grqq}

% Bulletpoints in Tabellen
\newcommand{\tabitem}{~~\llap{\textbullet}~~}

% TODO: Neue Seite für jede Section, da input statt include verwendet wird.
\newcommand{\sectionbreak}{\clearpage}

% Definition von \footref{#1: label}
% Verweis auf bereits existierende Fußnoten mittels
\providecommand*{\footref}[1]{
	\begingroup
		\unrestored@protected@xdef\@thefnmark{\ref{#1}}
	\endgroup
\@footnotemark}

% Definition von \mypageref{#1: label}
% Kombination aus \ref{#1: label} und \pageref{#1: label}
\newcommand{\mypageref}[1]{\ref{#1} auf Seite \pageref{#1}}

% Definition von \myboxquote{#1: text}
% grau hinterlegte Quotation-Umgebung (für Zitate)
\newcommand{\myboxquote}[1]{
	\begin{quotation}
		\colorbox{boxgray}{\parbox{0.78\textwidth}{#1}}
	\end{quotation}
	\vspace*{1mm}
}

\makeatletter
\zref@newprop*{appsec}{}
\zref@addprop{main}{appsec}

% Definition von \applabel{#1: label}{#2: text}
% von \appsec{#1: text}{#2: label} zur Erzeugung des Labels verwendet)
\def\applabel#1#2{%
	\zref@setcurrent{appsec}{#2}%   
	\zref@wrapper@immediate{\zref@label{#1}}%
}

% Definition von \appref{#1: label}
% anstelle \ref{#1: label} zum referenzieren von Anhängen verwenden)
\def\appref#1{%
	\hyperref[#1]{\zref@extract{#1}{appsec}}%
}
\makeatother

% Definition von \appsection{#1: text}{#2: label}
% Ersetzt \section{#1: text} und \label{#2: label} für Anhänge)
\newcommand{\theappsection}[1]{Anhang \Alph{section}:~\protect #1}
\newcommand{\appsection}[2]{
	\addtocounter{section}{1}
	\phantomsection
	\addcontentsline{toc}{section}{\theappsection{#1}}
	\markboth{\theappsection{#1}}{}

	\section*{\theappsection{#1}}
	\applabel{#2}{Anhang \Alph{section}}
	\label{#2}
}


%%%%%%%%%%%%% Index, Abkürzungsverzeichnis und Glossar erstellen %%%%%%%%%%%%
\makeindex
\makenomenclature
\makeglossaries

% Art der Zitierung (Havardmethode: Abkuerzung Autor + Jahr) %
\bibliographystyle{dinat}


%%%%%%%%%%%%%%%%%%%%%%%%%%%%%%% PDF-Optionen %%%%%%%%%%%%%%%%%%%%%%%%%%%%%%%%
\hypersetup{
	bookmarksopen=false,
	bookmarksnumbered=true,
	bookmarksopenlevel=0,
	pdftitle=\myTopic,
	pdfsubject=\myTopic,
	pdfauthor=\myAutor,
	pdfborder=0,
	pdfstartview=Fit,
	pdfpagelayout=SinglePage
}


%%%%%%%%%%%%%%%%%%%%%%%%%%%% Kopf- und Fußzeile %%%%%%%%%%%%%%%%%%%%%%%%%%%%%
\pagestyle{fancy}
\fancyhf{}
\fancyhead[R]{\thepage}
\renewcommand{\headrulewidth}{0.5pt}


%%%%%%%%%%%%%%%%%%%%%%%%% Layout und Beschriftungen %%%%%%%%%%%%%%%%%%%%%%%%%

% Zeilenabstand
\onehalfspacing
\setlist{noitemsep}

% Unterschriften für Tabellen und Abbildungen
\addto\captionsngerman{
  \renewcommand{\figurename}{Abb.}
  \renewcommand{\tablename}{Tab.}
}

% Numerierungen je Sektion zurücksetzen
\numberwithin{table}{section}
\numberwithin{figure}{section}

% Numerierungen mit Section
\renewcommand{\thetable}{\arabic{section}.\arabic{table}}
\renewcommand{\thefigure}{\arabic{section}.\arabic{figure}}

% Sektionsbezeichnung von Fußnoten entfernen
\renewcommand{\thefootnote}{\arabic{footnote}}

% Fussnoten durch Komma trennen
\renewcommand{\multfootsep}{, }


%%%%%%%%%%%%%%%%%%%%%%%%%%%%%%% Listingstyle %%%%%%%%%%%%%%%%%%%%%%%%%%%%%%%%
\lstset{
	basicstyle=\ttfamily\scriptsize,
	commentstyle=\color{commentgray}\textit,
	showstringspaces=false,
	stringstyle=\color{darkgreen},
	keywordstyle=\color{blue},
	numbers=left,
	numberstyle=\tiny,
	stepnumber=1,
	numbersep=15pt,
	tabsize=2,
	fontadjust=true,
	frame=single,
	backgroundcolor=\color{boxgray},
	captionpos=b,
	linewidth=0.94\linewidth,
	xleftmargin=0.1\linewidth,
	breaklines=true,
	aboveskip=16pt
}

%Seiten und Kapitel einbinden
\begin{document}
	\pagenumbering{Roman}
	% Wird automatisch ausgewählt.
	\ifcase\myType
		\begin{titlepage}
	\begin{center}
		\vspace*{2cm}
		\LARGE\bf\myTopic\\
		\Large\rm\mySubTopic\\
		\vspace*{3cm}
		\bf Seminararbeit zur Vorlesung\\
		\myVorlesung\\
		\normalsize\rm
		\vspace*{1cm}
		für die\\
		Prüfung zum Bachelor of Science\\
		\vspace*{1cm}
		an der Fakultät für Wirtschaft\\
		im Studiengang Wirtschaftsinformatik\\
		\vspace*{1cm}
		an der\\
		DHBW Ravensburg
		\vfill
	\end{center}
	\begin{tabular}{ll}
		Verfasser:&\myAutor\\
		Ausbildungsbetrieb:&\myCompany\\
		Anschrift:&\myCompanyAddressStreet\\
		&\myCompanyAddressCity\\
		Wiss. Betreuer:&\myProf\\
		Abgabedatum:&\myEndDate\\
	\end{tabular}
\end{titlepage}
\newpage
\setcounter{page}{2}

	\or
		\begin{titlepage}
	\begin{center}
		\vspace*{1cm}
		\LARGE\bf\myTopic\\
		\Large\rm\mySubTopic\\
		\vspace*{2cm}
		\bf \myProjNumber.~Projektarbeit\\
		\vspace*{1cm}
		\normalsize\rm
		Praxisphase des \myPraxPhase. Studienjahrs \\
		\vspace*{1cm}
		an der Fakultät für Wirtschaft\\
		im Studiengang Wirtschaftsinformatik\\
		\vspace*{1cm}
		an der\\
		DHBW Ravensburg
		\vfill
	\end{center}
	\begin{tabular}{ll}
		Verfasser:&\myAutor\\
		Ausbildungsbetrieb:&\myCompany\\
		Anschrift:&\myCompanyAddressStreet\\
		&\myCompanyAddressCity\\
		Wiss. Betreuer:&\myProf\\
		Abgabedatum:&\myEndDate\\
	\end{tabular}
	\newline
	\vspace*{1cm}
	\newline
	\begin{tabularx}{\textwidth}{l@{\extracolsep\fill}r}
	  Unterschrift des verantwortlichen Ausbilders&\\
	  (oder des Personalverantwortlichen)&\rule{6cm}{0.3mm}\\
	\end{tabularx}
\end{titlepage}
\newpage
\setcounter{page}{2}

	\or
		\begin{titlepage}
	\begin{center}
		\vspace*{2cm}
		\LARGE\bf\myTopic\\
		\Large\rm\mySubTopic\\
		\vspace*{3cm}
		\bf Bachelorarbeit\\
		\normalsize\rm
		\vspace*{1cm}
		für die\\
		Prüfung zum Bachelor of Science\\
		\vspace*{1cm}
		an der Fakultät für Wirtschaft\\
		im Studiengang Wirtschaftsinformatik\\
		\vspace*{1cm}
		an der\\
		DHBW Ravensburg
		\vfill
	\end{center}
	\begin{tabular}{ll}
		Verfasser:&\myAutor\\
		Ausbildungsbetrieb:&\myCompany\\
		Anschrift:&\myCompanyAddressStreet\\
		&\myCompanyAddressCity\\
		Wiss. Betreuer:&\myProf\\
		Abgabedatum:&\myEndDate\\
	\end{tabular}
\end{titlepage}
\newpage
\setcounter{page}{2}

	\else
	\fi
	
	\pagestyle{fancy}
	\begin{titlepage}
	\begin{center}
		\vspace*{1cm}
		\Huge\bf Sperrvermerk\\
		\vspace*{2cm}
		\normalsize\rm
		\begin{quotation}
			\parbox{0.8\textwidth}{Die vorliegende \ifcase\myType Seminararbeit \or Projektarbeit \or Bachelorarbeit\else\fi ~beinhaltet interne vertrauliche Informationen der Firma \myCompany, \myCompanyAddressStreet, \myCompanyAddressCity. Die Weitergabe des Inhaltes der Arbeit im Gesamten oder in Teilen ist grundsätzlich untersagt. Es dürfen keinerlei Kopien oder Abschriften - auch in digitaler Form - gefertigt werden. Ausnahmen bedürfen der schriftlichen Genehmigung der Firma \myCompany, \myCompanyAddressStreet, \myCompanyAddressCity.}
		\end{quotation}
		\vspace*{1cm}
		\begin{quotation}
		  \parbox{0.8\textwidth}{
		  \begin{tabularx}{0.78\textwidth}{l@{\extracolsep\fill}l}
				\rule{4cm}{0.3mm}&\rule{4cm}{0.3mm}\\
	    	Ort, Datum&Unterschrift
			\end{tabularx}}
		\end{quotation}
	\end{center}
\end{titlepage}
\newpage
\setcounter{page}{3}

	\tableofcontents
\newpage

	\renewcommand{\nomname}{Abkürzungsverzeichnis}
\setlength{\nomlabelwidth}{.25\hsize}
\renewcommand{\nomlabel}[1]{#1 \dotfill}
\setlength{\nomitemsep}{-\parsep}
\printnomenclature
\newpage

	\renewcommand{\glossaryname}{Glossar}
\printglossaries
	\listoffigures
\newpage

	\listoftables
\newpage

	%Listingnummering je Sektion zurücksetzen
\numberwithin{lstlisting}{section}
%Listingnummerierung mit Section
\renewcommand{\thelstlisting}{\arabic{section}.\arabic{lstlisting}}
%Listingsverzeichnis in das Inhaltsverzeichnis aufnehmen.
\renewcommand{\lstlistingname}{Listingsverzeichnis}
\renewcommand{\lstlistoflistings}{\begingroup
\tocchapter
\tocfile{\lstlistingname}{lol}
\endgroup}

\lstlistoflistings
\newpage

\renewcommand{\lstlistingname}{Listing}


	% Kapitel
	\pagestyle{fancy}
	\fancyhead[L]{\nouppercase{\leftmark}}
	\fancyhead[L]{\nouppercase{\rightmark}}
	\pagenumbering{arabic}
	\input{chapter/20_kapitel}
		
	% Anhang
	% Tabellennummerierung mit Section
	\renewcommand{\thetable}{\Alph{section}.\arabic{table}}
	% Abbildungsnummerierung mit Section
	\renewcommand{\thefigure}{\Alph{section}.\arabic{figure}}
	% Listingsnummerierung mit Section
	\renewcommand{\thelstlisting}{\Alph{section}.\arabic{lstlisting}}
	
	\begin{appendix}
	%\include{chapter/30_anhang}
	\end{appendix}
	
	% Abschluss
	\bibliography{literatur/literatur}
\newpage

	\renewcommand{\indexname}{Stichwortverzeichnis}
\printindex
\newpage

	\thispagestyle{empty}
\addcontentsline{toc}{section}{Selbständigkeitserklärung}
\begin{center}
	\vspace*{2cm}
	\Huge\bf Selbständigkeitserklärung\\
	\vspace*{3cm}
	\normalsize\rm
	Ich versichere hiermit, dass ich meine \ifcase\myType Seminararbeit \or Projektarbeit \or Bachelorarbeit\else\fi ~mit dem Thema\\
	\vspace*{2cm}
	\Large\bf\myTopic\\
	\Large\rm\mySubTopic\\
	\vspace*{2cm}
	\normalsize\rm
	selbständig verfasst und keine anderen als die angegebenen\\Quellen und Hilfsmittel benutzt habe.\\
	\vfill
	\begin{tabularx}{\textwidth}{l@{\extracolsep\fill}r}
  	\rule{7cm}{0.3mm}&\rule{7.55cm}{0.3mm}\\
	\end{tabularx}
	\begin{tabularx}{\textwidth}{*{2}{>{\arraybackslash}X}}
	  Ort, Datum&Unterschrift\\
	\end{tabularx}
\end{center}

\end{document}
